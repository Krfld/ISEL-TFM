\section{Motivation}

The motivation for this work is primarily to avoid ticket scalping. This is the process of buying tickets in bulk, exploiting high demand, and reselling them at significantly inflated prices. This not only disadvantages people that genuinely want to attend but also undermines the integrity of the ticketing system. Traditional ticketing platforms often rely on centralized databases and intermediaries, providing opportunities for scalpers to manipulate the system and engage in fraudulent activities. By using blockchain, it's possible to create a system that prevents any kind of unwanted operations, because of the properties of smart contracts, that enforce a certain behavior, allowing for users to resell a ticket, but not for a price higher than the original price. This is a way to guarantee that the end user will have a fair and transparent ticketing process.

For this to happen, the idea is to have a marketplace for users to resell their tickets, enforcing a price cap. With this feature, it's also possible to have a way for the organizer to promote the event on this platform, giving the possibility of reducing costs, by removing intermediaries like marketing companies.

Another important feature is the possibility of having partial refunds. This is something that is not always available in traditional ticketing platforms, and when it is, it's not always easy to do. With blockchain, it's possible to have a system that allows for easy and instant refunds, without the need for intermediaries. Enabling a feature like this can actually be advantageous for the organizers, if they're expecting the venue to be full. This way, when users buy their tickets and then realize they can't attend, they have a reason to refund, making that ticket available again for the original price, making a profit for the organizer.

Another point, and the main reason to use blockchain, is to guarantee the user of any operation. In the traditional ticketing system, there can be human errors, or even fraud, and this assures users that any operation defined will never change and will be executed as expected.