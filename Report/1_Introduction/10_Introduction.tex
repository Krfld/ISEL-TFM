\chapter{Introduction}

Concerts and festivals play a big role in peoples life's, allowing them to create memorable experiences watching live performances from their favorite artists. Those are the kinds of memories that last for life, so every event organizer wants to ensure that the whole process works seamlessly, from the end user to the entire background planning of the event.

The process of organizing an event starts with the event organizer, who is responsible for the whole planning of the event, from the venue to the artists, to the marketing and ticketing. The event organizer is the one who takes the risk of organizing the event, and the one who will profit from it, and can be a company, a group of people, or even a single person, whom will hire the artists, the venue, the security, the marketing, and the ticketing.

There is an issue with the ticketing process, which is the scalping. Scalping is the process of buying tickets in bulk, exploiting high demand, and reselling them at significantly inflated prices. This not only disadvantages people that genuinely want to attend but also undermines the integrity of the ticketing system. Traditional ticketing platforms often rely on centralized databases and intermediaries, providing opportunities for scalpers to manipulate the system and engage in fraudulent activities.
Moreover, existing ticketing systems frequently encounter issues related to security, trust, and reliability. Centralized databases are vulnerable to cyber attacks, leading to unauthorized access, data breaches, and the manipulation of ticketing information. Trust in the authenticity of tickets and the reliability of transactions is compromised, creating a pressing need for innovative solutions that can address these inherent challenges.

That's where blockchain comes in. Blockchain technology, renowned for its decentralized and transparent nature, presents a compelling solution to revolutionize the ticketing industry. By leveraging blockchain, it becomes possible to create a secure and tamper-proof ledger of transactions, mitigating the risk of scalping and ensuring the integrity of the ticketing process. The use of smart contracts further automates transactions, reducing the reliance on intermediaries, therefore extra costs, and enhancing operational efficiency. This allows the event organizer to have a more secure and reliable ticketing process, and the end user to have a more transparent and fair ticketing process.

\section{Motivation}

The motivation for this work is primarily to avoid ticket scalping. By using blockchain, it's possible to create a system that prevents any kind of unwanted operations because of the properties of smart contracts that enforce a certain behavior, allowing for users to resell a ticket, but not for a price higher than the original price. This is a way to guarantee that the end user will have a fair and transparent ticketing process. For this to happen, the idea is to have a marketplace where users are able to resell their tickets, enforcing a price cap on them.

Another important feature is the possibility of having partial refunds. This is something that is not always available in traditional ticketing platforms, and when it is, it's not easy to do. With blockchain, it's possible to have a system that allows for easy and instant refunds, without the need for intermediaries. Enabling a feature like this can actually be advantageous for the organizers, if they're expecting the venue to be full. This way, when users buy their tickets and then realize they can't attend, they have a reason to refund, making that ticket available again for the original price, making a profit for the organizer.

Another point, and the main reason to use blockchain, is to guarantee the user of any operation. In the traditional ticketing system, there can be human errors, or even fraud, and this assures users that any operation defined will never change and will be executed as expected.
\section{Objectives}

It will be crated a Ticketchain smart contract to have a place to manage organizers and their events. We could allow for anyone to create an event on this platform, but that would cause a lot of spam. This way, we have to allow the organizer to our system, and then they can create their events. For each event, there is its own smart contract that is deployed, transfering its ownership to the event organizer, so that Ticketchain can't interfere with any operation. We'll be having a website to ease the access to that smart contract and to allow for the organizer to manage their events.

Then, there will be an app for the users to check the events and manage their tickets. The events shown are gonna be the ones stored in our Ticketchain smart contract. They will aslo have access to the marketplace, where they will be able to resell their tickets for a price no higher than the original one.

For the validators, we will have another app that allows them to validate user tickets at the entrance of the venue. These validators are selected by the organizer on the event smart contract, and their job is to truthfully verify the tickets, without any chance of fraud. These validators can be either a person with their app to operate or even a turnstile.

\section{Contributions}

The repository for the frontend can be found on the GitHub in
\href{https://github.com/Krfld/Ticketchain}{Ticketchain frontend repository}.
The blockchain smart contracts code can be found on the
\href{https://github.com/Krfld/Ticketchain-Foundry}{Ticketchain backend
    repository} and it was deployed on the Base Sepolia Testnet at
\href{https://sepolia.basescan.org/address/0x87f4a5c17c2d3dc48f8e19d81e319230fa28f20d}{BaseScan}
so you can interact directly with.

\section{Document Structure}
\label{sec:document_structure}

The document is structured in the following way: \textbf{Chapter
    \ref{ch:introduction}: Introduction} provides an overview of the project,
including its context, motivation, objectives, and contributions;
\textbf{Chapter \ref{ch:background_and_related_work}: Background and Related
    Work} introduces essential concepts and technologies that serve as the
foundation for the project; \textbf{Chapter \ref{ch:requirements_analysis}:
    Requirements Analysis} outlines the functional and non-functional requirements
of the project as well as the use cases and the architecture; \textbf{Chapter
    \ref{ch:implementation}: Implementation} describes the implementation of the
project, including the technologies used; \textbf{Chapter \ref{ch:results}:
    Results} presents the results of the project and how it meets the requirements;
\textbf{Chapter \ref{ch:conclusions}: Conclusion} summarizes the project,
reflecting on the achievements and limitations, and suggesting future work.

Each chapter is divided into sections and subsections to provide a detailed
exploration of the project's various aspects. The document is supplemented with
figures, tables, and code snippets to enhance the reader's understanding of the
project.
