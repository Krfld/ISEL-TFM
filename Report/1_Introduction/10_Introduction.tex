\chapter{Introduction}

Concerts and festivals play a big role in peoples life's, allowing them to create memorable experiences watching live performances from their favorite artists. Those are the kinds of memories that last for life, so every event organizer wants to guarantee that the whole process works seamlessly, from the end user to the entire background planning of the event.

The process of organizing an event starts with the event organizer, who is responsible for the whole planning of the event, from the venue to the artists, to the marketing and ticketing. The event organizer is the one who takes the risk of organizing the event, and the one who will profit from it, and can be a company, a group of people, or even a single person, whom will hire the artists, the venue, the security, the marketing, and the ticketing.

There is an issue with the ticketing process, which is the scalping. Scalping is the process of buying tickets in bulk, exploiting high demand, and reselling them at significantly inflated prices. This not only disadvantages people that genuinely want to attend but also undermines the integrity of the ticketing system. Traditional ticketing platforms often rely on centralized databases and intermediaries, providing opportunities for scalpers to manipulate the system and engage in fraudulent activities.
Moreover, existing ticketing systems frequently encounter issues related to security, trust, and reliability. Centralized databases are vulnerable to cyber attacks, leading to unauthorized access, data breaches, and the manipulation of ticketing information. Trust in the authenticity of tickets and the reliability of transactions is compromised, creating a pressing need for innovative solutions that can address these inherent challenges.

That's where blockchain comes in. Blockchain technology, renowned for its decentralized and transparent nature, presents a compelling solution to revolutionize the ticketing industry. By leveraging blockchain, it becomes possible to create a secure and tamper-proof ledger of transactions, mitigating the risk of scalping and ensuring the integrity of the ticketing process. The use of smart contracts further automates transactions, reducing the reliance on intermediaries, therefore extra costs, and enhancing operational efficiency. This allows the event organizer to have a more secure and reliable ticketing process, and the end user to have a more transparent and fair ticketing process.

\section{Motivation}

The motivation for this work is primarily to avoid ticket scalping. This is the process of buying tickets in bulk, exploiting high demand, and reselling them at significantly inflated prices. This not only disadvantages people that genuinely want to attend but also undermines the integrity of the ticketing system. Traditional ticketing platforms often rely on centralized databases and intermediaries, providing opportunities for scalpers to manipulate the system and engage in fraudulent activities. By using blockchain, it's possible to create a system that prevents any kind of unwanted operations, because of the properties of smart contracts, that enforce a certain behavior, allowing for users to resell a ticket, but not for a price higher than the original price. This is a way to guarantee that the end user will have a fair and transparent ticketing process.

For this to happen, the idea is to have a marketplace for users to resell their tickets, enforcing a price cap. With this feature, it's also possible to have a way for the organizer to promote the event on this platform, giving the possibility of reducing costs, by removing intermediaries like marketing companies.

Another important feature is the possibility of having partial refunds. This is something that is not always available in traditional ticketing platforms, and when it is, it's not always easy to do. With blockchain, it's possible to have a system that allows for easy and instant refunds, without the need for intermediaries. Enabling a feature like this can actually be advantageous for the organizers, if they're expecting the venue to be full. This way, when users buy their tickets and then realize they can't attend, they have a reason to refund, making that ticket available again for the original price, making a profit for the organizer.

Another point, and the main reason to use blockchain, is to guarantee the user of any operation. In the traditional ticketing system, there can be human errors, or even fraud, and this assures users that any operation defined will never change and will be executed as expected.
\section{Objectives}

{\color{red}
\begin{itemize}
    \item develop smart contracts
    \item app for users to manage their tickets
    \item authentication logic on tickets to verify ticket ownership
    \item develop dashboard to allow event organizers to deploy their events
\end{itemize}
}

\section{Contributions}

 [website]
 [user app]
 [validator app]
\section{Document Structure}
\label{sec:document_structure}

The document is structured in the following way: \textbf{Chapter
    ~\ref{ch:introduction}: Introduction} provides an overview of the project,
including its context, motivation, objectives, and contributions;
\textbf{Chapter~\ref{ch:background_and_related_work}: Background and Related
    Work} introduces essential concepts and technologies that serve as the
foundation for the project; \textbf{Chapter~\ref{ch:requirements_analysis}:
    Requirements Analysis} outlines the functional and non-functional requirements
of the project as well as the use cases and the architecture; \textbf{Chapter
    ~\ref{ch:implementation}: Implementation} describes the implementation of the
project, including the technologies used; \textbf{Chapter~\ref{ch:results}:
    Results} presents the results of the project and how it meets the requirements;
\textbf{Chapter~\ref{ch:conclusions}: Conclusions} summarizes the project,
reflecting on the achievements and limitations, and suggesting future work.

