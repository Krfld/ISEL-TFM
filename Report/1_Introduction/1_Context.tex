\section{Context}
Concerts and festivals play a big role in peoples life's, allowing them to create memorable experiences watching live performances from their favorite artists. Those are the kinds of memories that last for life, so every event organizer wants to guarantee that the whole process works seamlessly, from the end user to the entire background planning of the event.

The process of organizing an event starts with the event organizer, who is responsible for the whole planning of the event, from the venue to the artists, to the marketing and ticketing. The event organizer is the one who takes the risk of organizing the event, and the one who will profit from it, and can be a company, a group of people, or even a single person, whom will hire the artists, the venue, the security, the marketing, and the ticketing.

There is an issue with the ticketing process, which is the scalping. Scalping is the process of buying tickets in bulk, exploiting high demand, and reselling them at significantly inflated prices. This not only disadvantages people that genuinely want to attend but also undermines the integrity of the ticketing system. Traditional ticketing platforms often rely on centralized databases and intermediaries, providing opportunities for scalpers to manipulate the system and engage in fraudulent activities.
Moreover, existing ticketing systems frequently encounter issues related to security, trust, and reliability. Centralized databases are vulnerable to cyber attacks, leading to unauthorized access, data breaches, and the manipulation of ticketing information. Trust in the authenticity of tickets and the reliability of transactions is compromised, creating a pressing need for innovative solutions that can address these inherent challenges.

That's where blockchain comes in. Blockchain technology, renowned for its decentralized and transparent nature, presents a compelling solution to revolutionize the ticketing industry. By leveraging blockchain, it becomes possible to create a secure and tamper-proof ledger of transactions, mitigating the risk of scalping and ensuring the integrity of the ticketing process. The use of smart contracts further automates transactions, reducing the reliance on intermediaries, therefore extra costs, and enhancing operational efficiency. This allows the event organizer to have a more secure and reliable ticketing process, and the end user to have a more transparent and fair ticketing process.


% ///
% The ticketing industry has long grappled with the challenge of ticket scalping, a practice wherein individuals purchase tickets in bulk, exploiting high demand, and resell them at significantly inflated prices. This not only disadvantages genuine event-goers but also undermines the integrity of the ticketing system. Traditional ticketing platforms often rely on centralized databases and intermediaries, providing opportunities for scalpers to manipulate the system and engage in fraudulent activities.

% Moreover, existing ticketing systems frequently encounter issues related to security, trust, and reliability. Centralized databases are vulnerable to cyber attacks, leading to unauthorized access, data breaches, and the manipulation of ticketing information. Trust in the authenticity of tickets and the reliability of transactions is compromised, creating a pressing need for innovative solutions that can address these inherent challenges.

% Blockchain technology, renowned for its decentralized and transparent nature, presents a compelling solution to revolutionize the ticketing industry. By leveraging blockchain, it becomes possible to create a secure and tamper-proof ledger of transactions, mitigating the risk of scalping and ensuring the integrity of the ticketing process. The use of smart contracts further automates transactions, reducing the reliance on intermediaries and enhancing operational efficiency.

% This research builds upon the increasing interest in blockchain applications across various industries and seeks to pioneer a transformative approach to ticketing. By exploring the potential of blockchain in eliminating scalping, enhancing security, and improving the reliability of ticket transactions, this thesis aims to contribute to the ongoing discourse on innovative technologies in the field of event management and ticketing. The outcomes of this research have the potential to reshape industry practices, establish new standards for transparency and security, and provide a model for the adoption of blockchain in other sectors requiring secure and trustful transactional systems.
