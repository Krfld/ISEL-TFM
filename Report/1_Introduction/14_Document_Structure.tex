\section{Document Structure}
\label{sec:document_structure}

The document is structured in the following way: \textbf{Chapter
    \ref{ch:introduction}: Introduction} provides an overview of the project,
including its context, motivation, objectives, and contributions;
\textbf{Chapter \ref{ch:background_and_related_work}: Background and Related
    Work} introduces essential concepts and technologies that serve as the
foundation for the project; \textbf{Chapter \ref{ch:requirements_analysis}:
    Requirements Analysis} outlines the functional and non-functional requirements
of the project as well as the use cases and the architecture; \textbf{Chapter
    \ref{ch:implementation}: Implementation} describes the implementation of the
project, including the technologies used; \textbf{Chapter \ref{ch:results}:
    Results} presents the results of the project and how it meets the requirements;
\textbf{Chapter \ref{ch:conclusions}: Conclusion} summarizes the project,
reflecting on the achievements and limitations, and suggesting future work.

Each chapter is divided into sections and subsections to provide a detailed
exploration of the project's various aspects. The document is supplemented with
figures, tables, and code snippets to enhance the reader's understanding of the
project.