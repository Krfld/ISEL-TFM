\section{Background}

\subsection{Blockchain}

Blockchain is a decentralized and distributed ledger technology that enables the secure recording and sharing of data across a network of computers. At its core, a blockchain consists of a series of blocks, each containing a list of transactions. These blocks are linked together in a chronological and immutable chain, forming a transparent and tamper-proof record of transactions.

~

Key characteristics of blockchain technology include:
\begin{itemize}
    \item \textbf{Decentralization}: Unlike traditional centralized systems where data is stored in a single location or controlled by a central authority, blockchain operates on a decentralized network of computers (nodes). Each node maintains a copy of the entire blockchain, ensuring that there is no single point of failure or control.
    \item \textbf{Transparency}: The data recorded on a blockchain is visible to all participants in the network. This transparency fosters trust among users, as they can independently verify the integrity of transactions and the state of the ledger without relying on intermediaries.
    \item \textbf{Immutability}: Once a transaction is recorded on the blockchain and added to a block, it becomes virtually impossible to alter or delete. This immutability is achieved through cryptographic techniques such as hashing and consensus mechanisms, ensuring that the historical record of transactions remains tamper-proof.
    \item \textbf{Security}: Blockchain technology employs advanced cryptographic algorithms to secure transactions and protect the integrity of the network. Transactions are verified and validated by network participants through a process known as consensus, which prevents fraudulent or unauthorized changes to the ledger.
    \item \textbf{Smart Contracts}: Smart contracts are self-executing contracts with predefined rules and conditions written in code. These contracts automate the execution of transactions and enforce agreements without the need for intermediaries. Smart contracts enable the creation of decentralized applications (DApps) that run on blockchain networks, facilitating a wide range of use cases beyond simple transaction processing.
\end{itemize}

Blockchain technology has applications across various industries, including finance, supply chain management, healthcare, and decentralized finance (DeFi). Its potential to revolutionize existing systems by enhancing security, transparency, and efficiency has led to widespread adoption and exploration of its capabilities in solving complex challenges.

\subsection{Smart Contracts}
Smart contracts are self-executing contracts with the terms of the agreement directly written into code. These contracts automatically execute and enforce themselves when predefined conditions are met, without the need for intermediaries such as lawyers or notaries. Smart contracts run on blockchain platforms, primarily Ethereum, and are stored and executed across a decentralized network of nodes.

~

Key characteristics of smart contracts include:
\begin{itemize}
    \item \textbf{Autonomy}: Once deployed on the blockchain, smart contracts operate autonomously, executing transactions and enforcing agreements without human intervention. This autonomy ensures that contract terms are upheld impartially and transparently.
    \item \textbf{Trust}: Smart contracts leverage the trustless nature of blockchain technology, meaning that parties can trust the execution of the contract without relying on a trusted third party. The decentralized and immutable nature of blockchain ensures that contract terms are tamper-proof and transparent.
    \item \textbf{Security}: Smart contracts are highly secure due to the cryptographic principles underlying blockchain technology. Once deployed, smart contracts cannot be altered or tampered with, providing a high level of security and reliability.
    \item \textbf{Efficiency}: By automating contract execution, smart contracts eliminate the need for intermediaries, reducing costs and processing times associated with traditional contract enforcement. Transactions are executed quickly and efficiently, enhancing the overall speed and efficiency of business processes.
    \item \textbf{Versatility}: Smart contracts can be programmed to execute a wide range of functions beyond simple transaction processing. They can facilitate complex conditional agreements, manage digital assets, and even interact with other smart contracts, enabling the development of decentralized applications (DApps) with diverse functionalities.
\end{itemize}

Smart contracts have numerous applications across various industries, including finance, supply chain management, real estate, healthcare, and more. They are particularly well-suited for scenarios where trust, transparency, and automation are paramount, offering a revolutionary approach to contract execution and enforcement in the digital age.

~

[mention networks]

[mention wallets]

[mention NFTs]
% [mention zk can store users data onchain, with the idea of avoiding a dedicated backend]
