\section{Background}

\subsection{Blockchain}

Blockchain is a decentralized and distributed ledger technology that enables the secure recording and sharing of data across a network of computers. At its core, a blockchain consists of a series of blocks, each containing a list of transactions. These blocks are linked together in a chronological and immutable chain, forming a transparent and tamper-proof record of transactions.

~

Key characteristics of blockchain technology include:
\begin{itemize}
    \item \textbf{Decentralization}: Unlike traditional centralized systems where data is stored in a single location or controlled by a central authority, blockchain operates on a decentralized network of computers (nodes). Each node maintains a copy of the entire blockchain, ensuring that there is no single point of failure or control.
    \item \textbf{Transparency}: The data recorded on a blockchain is visible to all participants in the network. This transparency fosters trust among users, as they can independently verify the integrity of transactions and the state of the ledger without relying on intermediaries.
    \item \textbf{Immutability}: Once a transaction is recorded on the blockchain and added to a block, it becomes virtually impossible to alter or delete. This immutability is achieved through cryptographic techniques such as hashing and consensus mechanisms, ensuring that the historical record of transactions remains tamper-proof.
    \item \textbf{Security}: Blockchain technology employs advanced cryptographic algorithms to secure transactions and protect the integrity of the network. Transactions are verified and validated by network participants through a process known as consensus, which prevents fraudulent or unauthorized changes to the ledger.
    \item \textbf{Smart Contracts}: Smart contracts are self-executing contracts with predefined rules and conditions written in code. These contracts automate the execution of transactions and enforce agreements without the need for intermediaries. Smart contracts enable the creation of decentralized applications (DApps) that run on blockchain networks, facilitating a wide range of use cases beyond simple monetary transactions.
\end{itemize}

Blockchain technology has applications across various industries, including finance, supply chain management, healthcare, and decentralized finance (DeFi). Its potential to revolutionize existing systems by enhancing security, transparency, and efficiency has led to widespread adoption and exploration of its capabilities in solving complex challenges. Some people refer to this ecosystem as Web3, which is a new paradigm for the internet that aims to decentralize control and empower users with greater ownership and privacy over their data and digital assets.
The reason a system like this works is because it leverages the human nature of greed and self-interest to create a system that is secure and reliable. The network of nodes is incentivized to maintain the integrity of the blockchain by rewarding them with cryptocurrency for their efforts. This creates a system where the majority of the network is honest and works together to maintain the integrity of the blockchain, making it resistant to attacks and fraud, as the cost of attacking the network would far outweigh any potential gains.

\subsection{Wallets}
Cryptocurrency wallets rely on cryptographic principles to securely manage and interact with digital assets on blockchain networks. These cryptographic techniques ensure the security and integrity of transactions while protecting the private keys that control access to cryptocurrency holdings.

~

Some of the key cryptographic aspects of wallets are:
\begin{itemize}
    \item \textbf{Private and Public Keys}: Cryptocurrency wallets utilize a pair of cryptographic keys: a public key and a private key. The public key, also known as the wallet address, is used to receive funds and is shared publicly. The private key, on the other hand, is known only to the wallet owner and is used to sign transactions and authorize the spending of funds. The relationship between the public key and the private key is based on asymmetric cryptography, where data encrypted with one key can only be decrypted with the other key. This ensures that transactions are secure and that only the rightful owner of the private key can access and control their cryptocurrency holdings.
    \item \textbf{Digital Signatures}: When a transaction is initiated from a cryptocurrency wallet, it is digitally signed using the wallet's private key. This digital signature serves as proof of authorization and ensures that the transaction cannot be tampered with or altered. Digital signatures are generated using cryptographic algorithms such as the Elliptic Curve Digital Signature Algorithm (ECDSA) or the Rivest-Shamir-Adleman (RSA) algorithm, depending on the specific blockchain network and protocol.
    \item \textbf{Hash Functions}: Cryptocurrency wallets use cryptographic hash functions to create a unique representation of transaction data, known as a transaction hash. These hash functions generate fixed-length strings of characters from input data, making it computationally infeasible to reverse-engineer the original data from the hash. Transaction hashes are essential for verifying the integrity of transactions and ensuring that they have not been altered or tampered with during transmission.
    \item \textbf{Seed Phrases and Mnemonic Codes}: Some cryptocurrency wallets use mnemonic codes or seed phrases as a backup mechanism for restoring access to wallet funds in case the original private key is lost or compromised. These seed phrases are generated from a random sequence of words and serve as a human-readable representation of the wallet's private key. They can be used to regenerate the private key and restore access to funds on a new wallet instance.
\end{itemize}

By leveraging these cryptographic techniques, cryptocurrency wallets provide a secure and reliable means for users to store, manage, and transact with digital assets on blockchain networks. The robustness of these cryptographic mechanisms ensures the confidentiality, integrity, and authenticity of transactions, safeguarding the value of cryptocurrency holdings against unauthorized access and fraudulent activities.

\subsection{Interacting with the Blockchain}

todo [explain what's needed to interact with the blockchain, with images]

~

As we saw, a blochchain works as a decentralized network of nodes, where each node has a copy of the entire blockchain. This means that in order to interact with the blockchain, we need to send transactions to the network, which will then be validated and added to a block by the nodes. To do this, we need to use a wallet, which is a software application that allows users to manage their digital assets, interact with smart contracts, and send transactions on the blockchain.
Wallets provide a user-friendly interface for accessing the blockchain network, signing transactions with private keys, and viewing account balances and transaction history. To execute a transaction on the blockchain, we need to sign it with our private key, which proves that we are the rightful owner of the assets being transferred. The transaction is then broadcast to the network, where it is validated by network participants and added to a block. Once the transaction is confirmed and included in a block, it becomes part of the immutable blockchain ledger, visible to all participants in the network.



\subsection{Networks}

This technology has evolved significantly since the inception of Bitcoin in 2009. Numerous platforms have emerged, each offering unique features, capabilities, and use cases.

~

Some of the most prominent networks that have gained traction in the decentralized ecosystem are:
\begin{itemize}
    \item \textbf{Bitcoin (BTC)}: Bitcoin is the first and most well-known cryptocurrency, introduced by an anonymous person or group of people under the pseudonym Satoshi Nakamoto in 2008. It operates on a decentralized network using a Proof of Work (PoW) consensus mechanism to validate transactions and secure the network. Bitcoin is designed as a peer-to-peer electronic cash system, enabling users to send and receive payments without the need for intermediaries. It has gained widespread adoption as a store of value and digital currency, with a fixed supply of 21 million coins and a deflationary monetary policy.
    \item \textbf{Ethereum (ETH)}: Ethereum is a decentralized, open-source blockchain platform that enables the creation and execution of smart contracts and decentralized applications (DApps). It introduced the concept of smart contracts, allowing developers to build a wide range of decentralized applications, from decentralized finance (DeFi) protocols to non-fungible token (NFT) marketplaces. Ethereum operates on a Proof of Work (PoW) consensus mechanism but is transitioning to a Proof of Stake (PoS) consensus model with the Ethereum 2.0 upgrade to improve scalability and energy efficiency.
    \item \textbf{Polygon (MATIC)}: Polygon is a Layer 2 scaling solution for Ethereum, designed to address the network's scalability issues by offering faster and cheaper transactions. It provides a framework for building and connecting Ethereum-compatible blockchain networks, known as sidechains, which leverage the security of the Ethereum mainnet. Polygon aims to enhance Ethereum's capabilities and support the mass adoption of
    \item \textbf{Solana (SOL)}: Solana is a high-performance blockchain platform designed for decentralized applications and crypto-currencies. It uses a unique combination of Proof of History (PoH) and Proof of Stake (PoS) consensus mechanisms to achieve high throughput and low latency, enabling fast transaction speeds and low fees. Solana aims to provide a scalable infrastructure for decentralized finance (DeFi), decentralized exchanges (DEXs), and other high-performance applications.
\end{itemize}

These are just a few examples of the diverse range of blockchain networks that exist, each offering unique features, capabilities, and use cases. As the blockchain ecosystem continues to evolve, new networks and technologies are constantly being developed, driving innovation and expanding the possibilities of decentralized applications and digital assets.

\subsection{Smart Contracts}
Smart contracts are self-executing contracts with the terms of the agreement directly written in code. These contracts automatically execute and enforce themselves when predefined conditions are met, without the need for intermediaries such as lawyers or notaries. Smart contracts run on blockchain platforms and are stored and executed across a decentralized network of nodes.

~

Key characteristics of smart contracts include:
\begin{itemize}
    \item \textbf{Autonomy}: Once deployed on the blockchain, smart contracts operate autonomously, executing transactions and enforcing agreements without human intervention. This autonomy ensures that contract terms are upheld impartially and transparently.
    \item \textbf{Trust}: Smart contracts leverage the trustless nature of blockchain technology, meaning that parties can trust the execution of the contract without relying on a trusted third party. The decentralized and immutable nature of blockchain ensures that contract terms are tamper-proof and transparent.
    \item \textbf{Security}: Smart contracts are highly secure due to the cryptographic principles underlying blockchain technology. Once deployed, smart contracts cannot be altered or tampered with, providing a high level of security and reliability.
    \item \textbf{Efficiency}: By automating contract execution, smart contracts eliminate the need for intermediaries, reducing costs and processing times associated with traditional contract enforcement. Transactions are executed quickly and efficiently, enhancing the overall speed and efficiency of business processes.
    \item \textbf{Versatility}: Smart contracts can be programmed to execute a wide range of functions beyond simple transaction processing. They can facilitate complex conditional agreements, manage digital assets, and even interact with other smart contracts, enabling the development of decentralized applications (DApps) with diverse functionalities.
\end{itemize}

Smart contracts have numerous applications across various industries, including finance, supply chain management, real estate, healthcare, and more. They are particularly well-suited for scenarios where trust, transparency, and automation are paramount, offering a revolutionary approach to contract execution and enforcement in the digital age.

\subsection{Token Standards}
Token standards play a crucial role in defining the rules and functionalities of digital tokens on blockchain networks. These standards provide a common framework that facilitates interoperability, compatibility, and ease of use for developers and users alike.

~

[add image]

~

Some of the most widely recognized token standards in the blockchain ecosystem are:
\begin{itemize}
    \item \textbf{ERC-20 (Ethereum Request for Comment 20)}: ERC-20 is the most commonly used token standard on the Ethereum blockchain, governing the creation and implementation of fungible tokens. These tokens are interchangeable and have identical properties, allowing them to be traded on cryptocurrency exchanges seamlessly. ERC-20 tokens adhere to a set of standard functions, including methods for transferring tokens, querying token balances, and approving token transfers on behalf of other addresses. Many of the initial coin offerings (ICOs), token sales, and decentralized finance (DeFi) projects on Ethereum utilize ERC-20 tokens due to their widespread adoption and compatibility with Ethereum wallets and exchanges.
    \item \textbf{ERC-721 (Ethereum Request for Comment 721)}: ERC-721 is a token standard on the Ethereum blockchain that governs the creation and implementation of non-fungible tokens (NFTs). Unlike ERC-20 tokens, each ERC-721 token is unique and indivisible, representing ownership or proof of authenticity of a specific asset. ERC-721 tokens are commonly used to represent digital assets such as digital art, collectibles, virtual real estate, and in-game items. Each token is assigned a unique identifier (token ID), allowing it to be distinguished from other tokens within the same contract. The ERC-721 standard defines methods for transferring tokens, querying token ownership, and managing metadata associated with each token, enabling a wide range of use cases in the burgeoning NFT market.
    \item \textbf{ERC-1155 (Ethereum Request for Comment 1155)}: ERC-1155 is a token standard on the Ethereum blockchain that supports the creation and management of both fungible and non-fungible tokens within the same contract. This allows developers to efficiently manage multiple token types and reduce gas costs associated with deploying multiple contracts. ERC-1155 tokens are highly flexible and versatile, making them suitable for a wide range of applications, including gaming, digital collectibles, and decentralized finance (DeFi). They provide developers with the ability to create tokenized assets with varying degrees of uniqueness and scarcity. The ERC-1155 standard defines methods for transferring tokens, querying token balances, and managing batch transfers of multiple token types, offering enhanced functionality compared to previous token standards.
\end{itemize}

These token standards represent just a few examples of the diverse range of standards shaping the landscape of tokenization on blockchain networks. As blockchain technology continues to evolve, new standards are likely to emerge, offering innovative solutions and driving further adoption of digital tokens across various industries.

\subsection{Non-Fungible Tokens (NFTs)}

Since we're talking about a ticketing system for events, we can see a lot of potential in the use of NFTs to represent digital tickets, providing a secure and verifiable means of ticket issuance, transfer, and validation. NFT-based tickets can be associated with unique metadata, such as event details, seat numbers, and access permissions, providing a rich and customizable ticketing experience for event organizers and attendees. NFTs can also be used to create limited edition or VIP tickets, offering exclusive access and additional benefits to holders of these special tickets.
