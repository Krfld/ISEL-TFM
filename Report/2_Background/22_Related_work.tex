\section{Related Work}

\subsection{Traditional Ticket Selling Platforms}

There are many different traditional ticket selling platforms that are used today. \emph{Ticketline} and \emph{Blueticket} are the largest and most reputable Portuguese companies specializing in ticket sales for a variety of events, including concerts, sports games, theater productions, and exhibitions. They also have a wide range of physical outlets, like \emph{Worten}, \emph{Fnac}, and \emph{El Corte Inglés}, making it convenient to buy tickets in person.
They have a website where they list all the events organizers are selling tickets for. The user can then open the event they are interested in and check all the details about it and it allows for the user to buy the tickets online and print them at home, or directs them to a physical outlet where they can buy them in person.
This means the organizers have to pay for these kinds of services to publish their event on those platforms and their management. This creates a big disadvantage for small organizers, who may not have the budget to cover all these costs.
That's where \emph{Ticketchain} comes in. The idea of developing the service on the blockchain ecosystem allows for a reduction in the costs because of its decentralized nature, along with the added security because a backend that needs to be maintained can have security vulnerabilities.

~

[show table of comparison between traditional ticket selling platforms and Ticketchain]

~

[ticketmaster]

\subsection{Application of NFTs}