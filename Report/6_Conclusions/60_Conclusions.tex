\chapter{Conclusions}
\label{ch:conclusions}

This project focused on the development of a decentralized ticketing system
that allows users to purchase, gift, refund, and validate tickets. Leveraging
blockchain technology, we deployed smart contracts on a testnet (test network),
which ensured transparency and security throughout the ticketing process.

The primary objective was to create a system that could be viable for
real-world use. We believe this goal has been met. The system is fully
operational and accessible to any user with a compatible wallet on the deployed
network. The system's scalability allows for the addition of multiple event
organizers, events, and users without compromising performance.

A key feature of the system is its ability to prevent ticket scalping by
utilizing blockchain's immutable nature. Additionally, the system ensures users
cannot be defrauded, as the ticket validation process confirms legitimate
ownership. Furthermore, the system enables validators to validate multiple
tickets simultaneously, making it possible for one user to purchase tickets for
a group.

We also incorporated a refund feature, enabling users to return tickets if they
are unable to attend an event. This is an uncommon feature in traditional
ticketing systems and could potentially benefit event organizers, as refunded
tickets can be resold at the original price.

\section{Limitations}
\label{sec:limitations}

Several limitations emerged during the development of this project, most of
which are tied to the current state of blockchain technology.

The primary limitation is network fees, which users must cover when performing
transactions, even if these fees are as low as 0.0001 euros. This poses a
particular challenge for organizers, who must account for the cost of
validators' operations. Though low fees accumulate slowly, they can become
significant depending on the number of tickets validated.

Another limitation is transaction finality—the time required for a transaction
to be confirmed on the blockchain. This impacts validators, who must wait for
confirmations before allowing entry to an event. Networks offering fast
finality often have high fees or limited scalability in terms of transactions
per second.

A well-known network that meets these criteria is Solana. However, its
incompatibility with the EVM presents another challenge, as it would require
rewriting the smart contracts in Rust.

Additionally, while users can gift tickets through the system, they could still
exchange tickets for money in person. While we encourage users to operate
within the system, we cannot control such external transactions. However,
patterns of frequent gifting could be flagged for potential scalping.

\section{Future Work}
\label{sec:future_work}

Several potential improvements were identified during this project, which could
enhance both user experience and system functionality.

Firstly, we developed a single mobile application combining user and validator
functionalities to streamline testing. As future work, we propose splitting
this into two separate applications: one for users and another for validators.
This would allow for a more user-friendly interface for general users and a
more robust interface for validators.

Another feature that was not implemented is a marketplace for ticket resale.
Currently, users can only buy, gift, or refund tickets. A marketplace would
enable users to resell tickets, giving others the opportunity to purchase
tickets for sold-out events. By capping resale prices, the marketplace could
also help combat scalping.

Additionally, we did not implement a web-based dashboard for event organizers.
This dashboard would allow organizers to create and manage events more easily,
automating the process of managing ticket metadata, which is currently handled
manually.

Further improvements could involve integrating web2 technologies alongside web3
features. For example, offering users the option to authenticate using email or
social media accounts, and abstracting the wallet connection process, would
improve user adoption by simplifying the blockchain interaction.

Lastly, incorporating fiat currency (e.g., euros) as a payment option,
alongside network-based currency, would provide users with a clearer
understanding of ticket prices. Additionally, implementing seat-specific
tickets, rather than just packages, would enhance the system's usability in
venues like stadiums or theaters, elevating the platform to a more professional
level.
