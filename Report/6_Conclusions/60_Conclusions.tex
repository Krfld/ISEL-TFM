\chapter{Conclusions}
\label{ch:conclusions}

In this project we developed a decentralized ticketing system, where users can
buy, gift, refund and validate tickets. We used blockchain technologies to
achieve this, and we used a testnet (test network) to deploy the smart
contracts.

The main goal of this project was to create a system that could be used in real
life, and we believe we achieved that. The system is fully functional, and it
can be used by anyone with a wallet that supports the network we deployed the
smart contracts on. The system is also scalable, as we can add as many
organizers and events as we want and have as many users as we want, and the
system will handle it.

We created a system that prevents scalping by leveraging the blockchain
technology. The system also prevents users from scamming others, as the
validation process guarantees that the owner really owns the tickets. Also the
validator can validate several tickets at once, so one user can buy for a group
of friends.

We also allow for a refund functionality, where users can refund their tickets
if they can't attend the event. This is a feature that is not usually available
in other ticketing systems, and the organizers might be able to profit event
more from this, as they can sell the ticket again at the original price.

\section{Limitations}
\label{sec:limitations}

There were of course some limitations found during the development of this
project. The main limitation is the fact that we are using blockchain
technologies, which are still in the early stages of development.

The main one would be the network fees. The network fees could be a problem,
but it all depends on the network. The users still have to pay for the fees of
a transaction, even if they are as low as 0.0001 euros.

This is specially a trouble for organizers, because they have to account for
the validators' operations to validate the tickets. Even if the fees are low,
it could accumulate to a significant amount of money, depending how many
tickets they validate. There are networks where this could not be a big problem
because the fees can really be as low as 0.0001 euros, but for those networks
maybe they don't match the rest of the requirements of the system, like the
finality, which is another limitation.

The finality is the time it takes for a transaction to be confirmed. This is
another limitation because the validators should wait for the validate
transaction to be confirmed before they allow users into the venue. Usually
when a network has good finality, either it has high fees, or its not scalable
(not enough transactions per second).

There is actually just one famous network that matches all these requirements,
and that's Solana. The problem is that this network is not EVM compatible, so
we would have to rewrite the smart contracts into Rust.

One thing to have in mind too is that users can still gift others in exchange
for money in person, but we have to let users know they should only operate
within the system. This is a limitation because we can't really control what
users do in person, but we would let them know that they should only operate
within the system. A good thing is that we can detect a pattern if for example
a user is always gifting tickets, it might mean that he's scalping, and that
could be investigated if necessary.

\section{Future Work}
\label{sec:future_work}

There are several things that weren't implemented in this project that could
improve the user experience and the overall functionality of the app. Due to
time constraints and to ease the testing process, we did only one mobile
application, where we merged the common users and validators functionalities.
So as for future work, we could separate that into two different applications,
one for the users and another for the validators. This way, we could have a
more user-friendly interface for the users and a more robust interface for the
validators.

The other goal that wasn't implemented was the marketplace. We could have a
marketplace where users could resell their tickets. At the moment, users can
only buy, gift and refund tickets, and it would be a great feature to have a
marketplace where users could resell their tickets. This way, users could buy
tickets for events that are sold out, and organizers could have a better idea
of the demand for their events. And avoiding, of course, the main problem of
ticket scalping, which can only be achieved if we have a dedicated marketplace
because only this way we can cap the prices of the tickets.

The last feature that wasn't implemented was the website dashboard for the
organizers, where they could create and manage their events. Currently that
would have to be done manually, or through a script. This dashboard would also
automate the process to handle the tickets metadata, instead of handling it
manually too.

There are a few extra things that could improve the system in the future. For
this project we aimed to use exclusively web3 technologies, and as mentioned,
this can improve the user experience by including the current web2
technologies, like authenticating with email and socials, abstracting the
wallet connection. This greatly improves user adoption, as users don't have to
worry about the blockchain.

In addition to this, there is a possibility to add functionality to the current
codebase to have the prices in a fiat currency like euro, not in the network
currency, so that users can have a better idea of the price of the tickets.

Another thing is that currently we don't have the logic to allow users to see
which ticket refers to each seat, and only just see them by packages. This
could be a complex feature to implement on the organizers side, but it would
step up the system to a more professional level, and could be used in stadiums
or even theaters.