\chapter{Implementation}
\label{ch:implementation}

[Brief introduction of the chapter]

\section{Mobile App}
\label{sec:mobile_app}

The mobile app is the main interface for the users to interact with the system,
where they can authenticate, see the events, and buy and manage tickets. It
will have a main page to check the events and a page for each event, where the
user can see the details and buy the tickets, and will also have a page to see
the tickets he owns.

We also made the validator logic in the same app to simplify the process, so we
don't have to make a separate app for the validators, however in a real
scenario we would have separate apps.

The app will be developed using the Flutter framework, which is a
cross-platform framework that allows us to build apps for Android and iOS from
a single codebase. It's made by Google and it's very popular for its ease of
use and performance.

\subsection{Authentication}
\label{subsec:authentication}

The Figure \ref{fig:authentication_page} shows the starting point of the app
that separates the authentication of the common users from the validators.

\begin{figure}[H]
	\includegraphics[width=\textwidth/3,frame]{Authentication page.jpg}
	\centering
	\caption{Authentication page}
	\label{fig:authentication_page}
\end{figure}

To authenticate the users, we will use the
\href{https://walletconnect.com/}{Wallet Connect} service, which supports a
variety of wallets to interact with. What this service does is establishes a
connection between the app and a wallet, so that when a function needs to be
called, the wallet receives the prompt and signs the transaction after the
user's approval.

The Figure \ref{fig:wallet_connect_prompt} shows what happens when we try to
authenticate as common users which triggers the Wallet Connect service. This
lists all the wallets that are supported by the service and the user can choose
the one he uses.

\begin{figure}[H]
	\includegraphics[width=\textwidth/3,frame]{Wallet connect prompt.jpg}
	\centering
	\caption{Wallet connect prompt}
	\label{fig:wallet_connect_prompt}
\end{figure}

After choosing one, the wallet app will open automatically and the user will
have to approve the connection, as the Figure \ref{fig:metamask_connect} shows.
We're using \href{https://metamask.io/}{MetaMask} as the external blockchain
wallet, which is the most popular blockchain wallet and it's available as a
browser extension and as a mobile app. To use it, you just have to create a
wallet there and you'll be all set to start interacting with the app.

\begin{figure}[H]
	\includegraphics[width=\textwidth/3,frame]{MetaMask connect.jpg}
	\centering
	\caption{MetaMask connect}
	\label{fig:metamask_connect}
\end{figure}

The authentication is a one-time process, so the user doesn't have to do this
every time he wants to interact with the app. Basically when a connection is
established, the next time the app tries to reconnect to the wallet, it will
skip the prompt.

\subsection{Events}
\label{subsec:events}

After the common user authenticates, he will be redirected to the main page of
the app, where he can see the events that are available. The Figure
\ref{fig:main_page} shows the main page of the app, where the user can see the
events and search for them.

\begin{figure}[H]
	\includegraphics[width=\textwidth/3,frame]{Main page.jpg}
	\centering
	\caption{Main page}
	\label{fig:main_page}
\end{figure}

When clicking on one of the events, the user will be redirected to the event
page. The Figure \ref{fig:event_page} shows the event page, where the user can
see the details of the event and the packages available.

\begin{figure}[H]
	\includegraphics[width=\textwidth/3,frame]{Event page.jpg}
	\centering
	\caption{Event page}
	\label{fig:event_page}
\end{figure}

The user sees the name, description, location, date, packages and even the
refund information. When the user taps on the package he wants to buy, the
prompt shown in the Figure \ref{fig:buy_tickets_prompt} appears, where the user
can choose the amount of tickets he wants to buy. We went with this approach of
only choosing the amount of tickets to buy, but in the future a feature like
seat selection could be implemented.

\begin{figure}[H]
	\includegraphics[width=\textwidth/3,frame]{Buy tickets prompt.jpg}
	\centering
	\caption{Buy tickets prompt}
	\label{fig:buy_tickets_prompt}
\end{figure}

The user can then confirm the purchase, and the wallet will prompt the user to
sign the transaction, as shown in the Figure
\ref{fig:metamask_transaction_prompt}. It displays the amount of money the user
has to pay, to which address he's interacting with, and the total cost to
execute the transaction.

\begin{figure}[H]
	\includegraphics[width=\textwidth/3,frame]{MetaMask transaction prompt.jpg}
	\centering
	\caption{MetaMask transaction prompt}
	\label{fig:metamask_transaction_prompt}
\end{figure}

\subsection{Tickets}
\label{subsec:tickets}

After confirmation, the user is redirected back to the app, where he will be
able to see the events which he owns any tickets, on the profile page, the
second tab with the profile icon, along with the button to disconnect from the
wallet, like shown in the Figure \ref{fig:profile_page}.

\begin{figure}[H]
	\includegraphics[width=\textwidth/3,frame]{Profile page.jpg}
	\centering
	\caption{Profile page}
	\label{fig:profile_page}
\end{figure}

Going into one of them, like the Figure \ref{fig:tickets_page} shows, the user
can see the tickets he owns for that event. It's a similar page as the normal
event one, but with the tickets he owns instead of the packages available. We
see 4 tickets here and the first one has a mark on it. This means the ticket
has already been validated. In the real world, a user do this doesn't make
sense, but we did it to make sure we handle every case.

\begin{figure}[H]
	\includegraphics[width=\textwidth/3,frame]{Tickets page.jpg}
	\centering
	\caption{Tickets page}
	\label{fig:tickets_page}
\end{figure}

Clicking on one of the tickets, it shows us the basic ticket information, along
with its image, like shown in the Figure \ref{fig:ticket_information}.

\begin{figure}[H]
	\includegraphics[width=\textwidth/3,frame]{Ticket information.jpg}
	\centering
	\caption{Ticket information}
	\label{fig:ticket_information}
\end{figure}

For operating the tickets, the user can simply click on the select tickets
button which will allow him to choose the tickets which he wishes to operate,
like shown in the Figure \ref{fig:ticket_operations}. In this case the user
sees only 3 tickets (while the Figure \ref{fig:tickets_page} shows 4) because
since the first is already validated, it's not possible to operate on it
anymore. We see the options to gift, refund and validate the tickets he
selected.

\begin{figure}[H]
	\includegraphics[width=\textwidth/3,frame]{Ticket operations.jpg}
	\centering
	\caption{Ticket operations}
	\label{fig:ticket_operations}
\end{figure}

The gift option will ask the user for the address to which he wants to gift the
tickets. After that, it will trigger the wallet to sign the transaction, and
the tickets will be transferred to the address.

The refund option will ask the user to confirm the refund, and trigger the
wallet to confirm the transaction in which the tickets will be burned, making
them available again for other users to buy, and returning the correct amount
of money to the user.

The validate option will start the validation process, which will be explained
in the Section TODO.

% 
% 

\section{TODO Validator App}
\label{sec:validator_app}

For the validator app, we will have a simple interface for the validators to
execute the process mentioned in the Figure \ref{fig:ticket_validation}.

[TODO mention the wallet being local]

%  The app itself will behave as a wallet in this case, to make things simple. 
% So basically the app will generate a private key and when the validator validates the tickets, the app will sign the message with the private key and send it to the blockchain. Since this is done by our system, when the validator has the app, the organizer will need to pass him some funds

Since the validators need to sign the message, the organizer just has to make
sure the validators have enough funds to pay for the transaction.

\subsection{Ticket Validation}
\label{subsec:ticket_validation}

For the ticket validation, we must take into consideration a lot of aspects,
because it's not just checking if the user address has a ticket associated to
him. This is because, since the data is on the blockchain, anyone can see the
addresses where each ticket belongs to, and pretend he's the owner of the
ticket. For this to be secure, we need to guarantee the user is actually the
owner of the address, and here is where the cryptographic message signature
comes in, the same process that happens when executing a normal blockchain
transaction.

We came up with the process shown in the Figure \ref{fig:ticket_validation},
that shows the steps between the actors and the system to validate the tickets.

\begin{figure}[H]
	\includegraphics[width=\textwidth*2/3]{Ticket validation.png}
	\centering
	\caption{Ticket validation}
	\label{fig:ticket_validation}
\end{figure}

Here we see that the user reads the QR with a generated message from the
validator, signs it with his wallet, and generates a JSON with the signature
and useful information like the tickets to validate, the event, and the user
address. After the validator reads the JSON in the QR code, he checks if the
parameters match the ones on the blockchain, gets the address using a
cryptographic method to recover it through the original message and it's
signature, and checks if the user is the owner of the tickets. If everything
matches, the validator will trigger a transaction to mark the tickets as
validated, to avoid people sharing the accounts and using the same ticket
multiple times.

Both parties need to know the original message, so it matches. We could just
use a default message for everyone, so the users would just need to give the
signature to the validator, but this could become a security issue in case the
signature gets leaked, because anyone who has it could pretend to be a
different address. The idea here is to have a unique message for each user at
the time of the event, so it forces the user to sign it.

All this is reduced to a single transaction on the blockchain because,
depending on the network, the finality of a transaction (the time it takes for
a transaction to be fully registered on the blockchain) can take a while. This
was planned to be done in a single transaction to avoid congestion at the
entrance of the venue, so the users can enter the event with the least amount
of delay. This is an important aspect to consider when designing the system to
fullfil the system being fast, so like mentioned in the Section
\ref{subsec:non_functional_requirements}, the network choice is crucial.

\section{Main Smart Contract}
\label{sec:main_smart_contract}

The deployment will be done using \href{https://book.getfoundry.sh/}{Foundry},
a toolkit for the development and deployment of the smart contracts, which
makes it easy to deploy to any network, and also to test the contracts locally.

So smart contracts are similar to C++ mainly because it lies on a class-like
structure with variables to store data and methods, where the main difference
is that a class is called a contract and you can extend others to integrate
their functionalities. That's essentially what's gonna happen with each event,
extending the ERC721 standard, making it a collection of NFTs, where each NFT
is a ticket. Since this is the behavior we want (each event being a NFT
collection), we will have to deploy (instantiate, in C++) a new contract for
each event, so we will have a main contract to keep track of these events.

With this in mind, we created the Ticketchain main smart contract UML, like the
Figure \ref{fig:system_uml} shows. This contract will track the organizers and
the events associated with the system, along with the method to register a new
event with the necessary data, restricted to only organizers (to avoid
unauthorized people to interact).

\begin{figure}[H]
	\includegraphics[width=\textwidth]{Ticketchain UML.png}
	\centering
	\caption{Main smart contract UML (simplified)}
	\label{fig:system_uml}
\end{figure}

With this structure, since we have this main contract where all the events of
the system are stored, we can simply make a call to get them all, showcasing
them in the app's home page for users to search. Any event that is deployed
outside the system or if it gets removed from there, it won't be shown to the
users.

\section{Event Smart Contract}
\label{sec:event_smart_contract}

For the event contract, we'll be extending the ERC721 standard and adding the
necessary methods to interact with the tickets, like buying, selling, and
validating them. The reason to extend this standard and not implement the logic
manually is because it makes it compatible with the most common marketplaces
for NFTs, which allows for users to do what they desire with them after the
event. It also has the necessary methods to manage the tickets, like
transferring them between users, and the necessary operations to track these
operations.

\subsection{ERC721 Structure}
\label{subsec:erc721_structure}

The standard was obtained through the
\href{https://docs.openzeppelin.com/contracts/api/token/erc721#ERC721}{OpenZeppelin}
library, which is a collection of secure and community-vetted smart contracts
that are used by many projects in the Ethereum ecosystem. This library is a
great resource for developers to build secure and reliable smart contracts.

Analyzing its source code [TODO ref], and looking into the most important
variables and methods of the standard shown in the Figure \ref{fig:erc721_uml},
we can understand that the NFTs are simply a mapping of the token ID to the
owner address, so when you execute a transaction to get a token (this process
is called minting), the token ID is then associated to your address. Then for
each token it's possible set a URI, which is a link to the NFTs metadata,
usually being a JSON file with the necessary information about the token, like
the name, description, and image.

This link could point to anything, for example a google drive file, but the
common thing is to store the metadata on the IPFS, which is a decentralized
storage system, so the metadata is not stored on the blockchain itself
(onchain), which would be very expensive, but rather on a decentralized storage
system (offchain), which is much cheaper.

\begin{figure}[H]
	\includegraphics[width=\textwidth*2/3]{ERC721 UML.png}
	\centering
	\caption{ERC721 UML (simplified)}
	\label{fig:erc721_uml}
\end{figure}

The function \textit{tokenURI} is the one that is called by default in the
marketplaces to get the NFT's metadata, being one of the main reasons to extend
the ERC721 standard, because it enforces the implementation of this method. In
the Figure \ref{fig:erc721_uml} we see that it has the \textit{virtual}
keyword, meaning this can be overridden by the contracts that extend it, to
manipulate the way to store the metadata. We'll be mentioning this again in the
Section \ref{fig:package_logic}, about how the packages logic is implemented.

\subsection{Event Behavior}
\label{subsec:event_behavior}

So the event will be deployed and we need a certain control over the tickets.
One of the aspects we need to account for is that when deploying an event, and
since the event will extend the ERC721, any public functions on that standard
will be possible to execute. This is a problem because we don't want the users
to mint tickets whenever they want or transfer them between themselves from
outside the system, so we need to restrict these operations. As we saw already
on the Figure \ref{fig:erc721_uml}, only the \textit{safeTransferFrom} method
is public, so users could transfer NFTs between each other. We want that to be
possible, just not from outside the system, since that can lead users to
exploit the system and scalping the tickets easily. The minting, however, won't
be an issue because it's an internal method, so we will access it from the buy
method in the event and restrict it there.

The Figure \ref{fig:event_lifecycle} shows the lifecycle of the event, and what
restrictions are in place for the ticket operations. The dates above the line
indicate the states of the event, and below a small description of the
operations that are allowed when each state is reached.

\begin{figure}[H]
	\includegraphics[width=\textwidth]{Event lifecycle.png}
	\centering
	\caption{Event lifecycle}
	\label{fig:event_lifecycle}
\end{figure}

We will have 4 main states for the event after it has been registered (the ones
in bold), being the \textit{Open}, \textit{No refund}, \textit{Start}, and
\textit{End} dates. Once the event is registered, it will show up in the app
for user to see, and the organizers can set a later open date to allow ticket
minting (buying).

\paragraph{Open Date:} Once it hits the \textit{Open} date, we will allow the users to buy tickets,
which will mint the NFTs by executing the \textit{safeMint} method of the
ERC721 contract. We will allow users to operate over the NFTs, but only within
the system. If they try to call the \textit{safeTransferFrom} directly from the
ERC721 standard, it will revert, because we detect it's not being called
through the system.

\paragraph{No Refund Date:} After the \textit{No refund} date, we will prevent the users to call the refund
method, which essentially \textit{burns} the NFTs, removing them from the user
and making them available again. This is a nice operation to add because it
allows the users to get their money back if they can't attend the event. The
organizer decides the percentage of the refund and the deadline, which is there
to prevent users to buy a big amount of tickets and then refund them last
minute, which would be a way to exploit the system (in case of a 100\% refund,
they wouldn't risk anything). The other good thing for the organizer is when
the event is expected to be sold out. Since the users will get some money back,
they will have a reason to refund their tickets if they cannot attend the event
anymore, making them available again for other users to buy at the original
price, making the organizer a higher profit. After this deadline, the only
that'll be allowed is for users to resell their tickets in the system's
marketplace, which them to sell at a higher price than the refund (but never
higher than the original, of course).

\paragraph{Start Date:} The \textit{Start} date is there to tell the users when the event starts, so
basically when the gates will open. That's the date that appears in the app, so
the users know when to show up.

\paragraph{End Date:} The \textit{End} date tells when the event is over, unlocking all the ticket
operations to outside the system. So users can simply keep the tickets as a
souvenir or sell them in any marketplace, without any restrictions on the
tickets, including the removal of the price cap.

With this behavior in mind, we came up with the Event UML, like shown in the
Figure \ref{fig:event_uml}, where we added the necessary methods for the
organizer/admins to manage the event and the users to handle the tickets, which
then trigger the corresponding methods of the ERC721 contract. We added a
possibility to have admins so the organizer can distribute the workload of
executing the necessary operations to people he trusts.

\begin{figure}[H]
	\includegraphics[width=\textwidth*3/4]{Event UML.png}
	\centering
	\caption{Event smart contract UML (simplified)}
	\label{fig:event_uml}
\end{figure}

As we can see, the \textit{update} method has been overridden from the ERC721
standard to restrict the interactions with the NFTs according the defined
behavior. This method is the one that gets called anytime there's an operation
on any NFT, so we can implement here the necessary logic to restrict the
operations, and we can visualize that in the following flowchart, shown in the
Figure \ref{fig:nft_flowchart}.

\begin{figure}[H]
	\includegraphics[width=\textwidth*2/3]{NFT flowchart.png}
	\centering
	\caption{NFT flowchart}
	\label{fig:nft_flowchart}
\end{figure}

This logic is possible to implement because of modifiers, mentioned before.
What we essentially do on that modifier, is set a variable to true (meaning the
NFT operation is coming from within the system), execute the function it's
assigned to, and then we set the variable back to false. Then we apply that
modifier on the necessary functions, which we can see them with the operator
	{internalTransfer} in the event UML, shown in the Figure
\ref{sec:event_smart_contract}.

\subsection{Structs}
\label{subsec:structs}

As is possible to see in the Figure \ref{fig:event_uml}, and also in the Figure
\ref{fig:system_uml}, we have a few custom structs to organize better the data.
These structs are the \textit{Percentage}, \textit{EventConfig},
\textit{NFTConfig}, \textit{PackageConfig} and \textit{TicketchainConfig}
structs.

\paragraph{Percentage Struct:} The \textit{Percentage} struct is necessary because in Solidity there are no
floating point numbers, so we need a way to make calculations with percentages.
What this struct does is it stores the value of the percentage and the amount
of decimals it has, so if we want to calculate 55.50\% of a number, we would
have 555 as the value with 1 decimal, or 5550 with 2 decimals.

The struct is as follows:
\begin{lstlisting}[caption=Percentage struct]
    struct Percentage {
        uint256 value;
        uint256 decimals;
    }
\end{lstlisting}
so to obtain a percentage of some $x$ number, we do $y = \frac{x \times
		\text{Percentage.value}}{100\mathrm{e}^\text{Percentage.decimals}}$.

When working with ether units, it can be common to have values like 0.00005
ether, but it's rather rare to have small values in wei, like 1000 wei, so
applying this formula won't lose much precision (note that 1 ether is $10^{18}$
wei).

\paragraph{TicketchainConfig Struct:} The \textit{TicketchainConfig} struct is simply to keep it stored the system
address and the system fee percentage, so we can easily access this information
when applying the fees and withdrawing them, and is as follows:
\begin{lstlisting}[caption=TicketchainConfig struct]
    struct TicketchainConfig {
        address ticketchainAddress;
        Percentage feePercentage;
    }
\end{lstlisting}

\paragraph{NFTConfig Struct:} The \textit{NFTConfig} struct is just to store the NFTs basic information, like
the name, symbol, and base URI, to ease the input of the NFTs information when
registering the event:
\begin{lstlisting}[caption=NFTConfig struct]
    struct NFTConfig {
        string name;
        string symbol;
        string baseURI;
    }
\end{lstlisting}
The is the name of the NFT collection and the symbol is the abbreviation of it,
like the name being 'Ticketchain' and the symbol being 'TCK', for example. The
base URI is the link to the metadata of the NFTs, which will be used to get the
information about the tickets.

\paragraph{EventConfig Struct:} The \textit{EventConfig} struct is to store the event's entire configuration,
like the name, description, location, dates, and refund, like this:
\begin{lstlisting}[caption=EventConfig struct]
    struct EventConfig {
        string name;
        string description;
        string location;
        uint256 openDate;
        uint256 noRefundDate;
        uint256 startDate;
        uint256 endDate;
        Percentage refundPercentage;
    }
\end{lstlisting}

\paragraph{PackageConfig Struct:} Lastly, the \textit{PackageConfig} struct is there to store each package
information, to keep track of the ones that are available for the event:
\begin{lstlisting}[caption=PackageConfig struct]
    struct PackageConfig {
        string name;
        string description;
        uint256 price;
        uint256 supply;
        bool individualNfts;
    }
\end{lstlisting}
This structure will be better discussed in the next Section
\ref{subsec:ticket_packages}.

\subsection{Ticket Packages}
\label{subsec:ticket_packages}

It's common to see events with different types of tickets, like VIP, standard,
or even 3-day passes, each with its own price and benefits. We want to
implement this feature in the system, so we can have a better control over the
tickets and the users can choose the one that fits them better.

For that, we will allow the organizer to add packages, indicating the supply of
each one, and as we saw already, the NFTs are a mapping of the ID to the owner,
so we can organize the packages as a list, where the supply and order of them
assigns the ID of each NFT to the package, like the Figure
\ref{fig:package_logic} illustrates.

\begin{figure}[H]
	\includegraphics[width=\textwidth*2/3]{Package logic.png}
	\centering
	\caption{Package logic}
	\label{fig:package_logic}
\end{figure}

This way, whenever we need to get a ticket for a certain package, we can go
through the packages and see which one the ID is in. One only limitation with
this is if the event is already open (users can buy tickets), the only thing we
can allow the organizer to do is to add packages, neither remove or change
their order, because that would change the package associated to the tickets,
which would be a problem for the users that already bought them.

Now we just have to make sure the information obtained with the
\textit{tokenURI} method corresponds to the ticket, according to its package.
For this we will have a different metadata file for each package, with the
necessary information about the tickets. The \textit{individualNfts} boolean in
the \textit{PackageConfig} struct is there to indicate if the organizer wants
each ticket on the package to have its own metadata, or if they can share it.

According to this, the \textit{tokenURI} will return an URI like
'baseURI/packageId/ticketId' for a package with individual NFTs, and
'baseURI/packageId' for a package with shared metadata. Like this, when we
store the metadata on the IPFS, we store the metadata for each ticket inside a
folder of the packages with individual NFTs, and only a metadata file for each
package without individual NFTs.

\subsection{Metadata Storage}
\label{subsec:metadata_storage}

To store the NFTs metadata files, we'll be using the
\href{https://ipfs.tech/}{InterPlanetary File System (IPFS)} for storing the
NFTs metadata. The IPFS is a decentralized storage system where the data is
stored in a distributed network of nodes (decentralized), making it very secure
and reliable. This is a great solution for storing the metadata of the NFTs
because it's very cheap and easy to use, and it's a common practice in the
blockchain ecosystem.

Other options would be to store the data on some kind of server, but that would
be more expensive to maintain, and since we are dealing with NFTs, it's good
practice to store the data in a decentralized manner, to avoid any kind of
alteration on its contents, if the tickets possibly become valuable
collectibles.

This kind of issue was something that has happened before, where people bought
NFTs with the idea of them being somewhat valuable, but then the owner changed
the contents of the metadata, executing what was called of a \textit{rug pull},
which is a scam that made the NFTs worthless, keeping the money for himself.

To store the data on the IPFS, we will be using the
\href{https://www.pinata.cloud/}{Pinata} service, which does the heavy lifting
for interacting with the storage itself. To accomplish this, we just need to
arrange the files according to the ticket packages, like mentioned before in
the Section \ref{subsec:ticket_packages}, and then upload them to the IPFS,
getting the link to the metadata, which we'll then store on the contract as the
base URI. The files would be stored like the Figure \ref{fig:metadata_storage}
shows, being the packages 1 and 3 with shared metadata, and the package 2 with
individual NFTs.

\begin{figure}[H]
	\includegraphics[width=\textwidth]{Metadata storage.png}
	\centering
	\caption{Metadata storage}
	\label{fig:metadata_storage}
\end{figure}

\subsection{System Fees}
\label{subsec:system_fees}

One of the most important aspects of a system like this is the business model
we have to take into account. Since this is a service we want to deploy for
event organizers, we need to make this sustainable and profitable. This kind of
service aims to do some heavy lifting, with its own features, so we could set a
fee lower than the usual on the traditional marketplaces and ticket selling
platforms, since the organizers need to pay for each service.

These low fees are possible because with the system being deployed on the
blockchain, it stays there while the network is running, so the only extra cost
are the network fees when interacting with the event. For the users, each
interaction is paid by them, so when a user buys a ticket, the only thing to
take into account are the network fees, which depending on the network, can be
super low.

The other kind of fee the organizer needs to look out for is for the validators
to validate the tickets, which is a necessary operation to avoid people from
exploiting the system. These fees are paid by the validators, which the
organizer essentially manages, so we need to take this into account when
setting the system fee, to make it sustainable for the organizer to use our
system.

We'll set a fee on the main smart contract, where will be stored in the event
when registering it, so that if we decide to change it, the previous events
aren't affected. This is also because we want to abstract the user of any extra
fee, so the price the organizer sets, is the price the user pays, and the
system fee is taken from the tickets price. In case an event gets cancelled, or
a user decides to get a refund, the ticket fee is returned to the user
(proportional to the refund), making the system less profit, but guarantees the
users of a fair process. With this, we need to restrict the system to only
withdraw any profit after the event is over. Since this is rather an uncommon
case, the less profit that the system makes compensates for the trust that the
users and organizers will have on it.
