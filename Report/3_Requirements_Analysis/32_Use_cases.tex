\section{Use Cases}
\label{sec:use_cases}

This section presents the use cases for the system, categorized by four key
actors: the Ticketchain Owner in the Section \ref{subsec:ticketchain_owner}
where we detail the use cases for the Ticketchain owner, focusing on system
settings management and the oversight of event organizers; the organizer in the
Section \ref{subsec:organizer} that covers the use cases for organizers,
including event creation, management, and validator selection; the Validator in
the Section \ref{subsec:validator} where we describe the validator's use case,
which involves ticket validation to ensure that only users with valid tickets
can enter events; and the Common User in the Section \ref{subsec:common_user}
where we outline the use cases for common users, such as purchasing, gifting,
refunding, and reselling tickets. Each actor shares a common use case:
authentication, which is responsible for verifying user identities within the
system.

\subsection{Ticketchain Owner}
\label{subsec:ticketchain_owner}

As illustrated in Figure \ref{fig:ticketchain_owner_use_cases}, the Ticketchain
owner has the specific use case of managing event organizers.

\begin{figure}[H]
    \centering
    \includegraphics[width=0.5\textwidth]{Ticketchain owner use cases.png}
    \caption{Ticketchain Owner Use Cases}
    \label{fig:ticketchain_owner_use_cases}
\end{figure}

This control is essential to ensure that only authorized individuals have
access to the system. Additionally, the Ticketchain owner can manage system
settings, which are crucial for tailoring the system’s behavior to meet the
needs of organizers.

\subsection{Organizer}
\label{subsec:organizer}

As shown in Figure \ref{fig:organizer_use_cases}, the organizer has several key
use cases, including event creation and management.

\begin{figure}[H]
    \centering
    \includegraphics[width=0.5\textwidth]{Organizer use cases.png}
    \caption{Organizer Use Cases}
    \label{fig:organizer_use_cases}
\end{figure}

The organizer also oversees the selection of validators for each event,
ensuring that only authorized personnel can validate tickets. Furthermore, the
organizer can manage event settings, such as updating information or canceling
events when necessary.

\subsection{Validator}
\label{subsec:validator}

For validators, as illustrated in Figure \ref{fig:validator_use_cases}, the
primary use case is to validate users' tickets, allowing entry to the event.

\begin{figure}[H]
    \centering
    \includegraphics[width=0.5\textwidth]{Validator use cases.png}
    \caption{Validator Use Cases}
    \label{fig:validator_use_cases}
\end{figure}

This step is critical for maintaining security and ensuring that only users
with valid tickets can participate.

\subsection{Common User}
\label{subsec:common_user}

Figure \ref{fig:common_user_use_cases} presents the use cases for common users.

\begin{figure}[H]
    \centering
    \includegraphics[width=0.5\textwidth]{Common user use cases.png}
    \caption{Common User Use Cases}
    \label{fig:common_user_use_cases}
\end{figure}

Users can purchase tickets, gift them to others, request refunds (depending on
event policies), and resell tickets (with the condition that they cannot sell
at a price higher than the original). These use cases provide users with the
flexibility to manage their tickets according to their preferences while
adhering to system regulations.
