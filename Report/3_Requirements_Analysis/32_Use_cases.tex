\section{Use Cases}
\label{sec:use_cases}

This Section will present the use cases for the system, divided into four
actors: Ticketchain owner, organizer, validator, and user. All of them have one
similar use case, the authenticate use case, which is responsible for
authenticating the users on the system. On the Section
\ref{subsec:ticketchain_owner} we have the use cases for the Ticketchain owner,
where it mentions the management of the system settings and the control of the
organizers that have access to the system. On the Section
\ref{subsec:organizer} we have the use cases for the organizer that mention the
creation and management of the events, along with the control of the validators
for the event. On the Section \ref{subsec:validator} we have the use cases for
the validator, and the only thing he can do is to validate the users' tickets.
Lastly, on the Section \ref{subsec:user} we have the use cases for the common
user, like the purchase, gift, refund and resell of tickets.

\subsection{Ticketchain owner}
\label{subsec:ticketchain_owner}

As we can see in the Figure \ref{fig:ticketchain_owner_use_cases}, the
Ticketchain owner has the specific use case to control event organizers. This
is important because the he needs to have control over the organizers that have
access to the system. The Ticketchain owner can also manage the system
settings, which is important to control the system's behavior and to adapt it
to the organizers' needs.

\begin{figure}[H]
    \includegraphics[width=\textwidth/2]{Ticketchain owner use cases.png}
    \centering
    \caption{Ticketchain owner use cases}
    \label{fig:ticketchain_owner_use_cases}
\end{figure}

\subsection{Organizer}
\label{subsec:organizer}

The organizer has the use cases to create events, as we can see in the Figure
\ref{fig:organizer}. He can also control the validators for the event, which is
important to select the people that have the authority to validate the tickets
for each event. The organizer can also manage the event settings, like updating
the event information or cancel it if really needed.

\begin{figure}[H]
    \includegraphics[width=\textwidth/2]{Organizer use cases.png}
    \centering
    \caption{Organizer use cases}
    \label{fig:organizer_use_cases}
\end{figure}

\subsection{Validator}
\label{subsec:validator}

For the validators, as we see in the Figure \ref{fig:validator_use_cases}, the
only use case is to validate the users' tickets and to allow them to enter the
event. This is a necessary step to avoid users to try to bypass this security
measure and to ensure that only the users that have a valid ticket can enter
the event.

\begin{figure}[H]
    \includegraphics[width=\textwidth/2]{Validator use cases.png}
    \centering
    \caption{Validator use cases}
    \label{fig:validator_use_cases}
\end{figure}

\subsection{User}
\label{subsec:user}

In the Figure \ref{fig:user_use_cases} we see the use cases for the common
user. The user can purchase tickets for the events, gift tickets to other
users, refund tickets if he doesn't want to go to the event anymore (depending
on the configuration of the specific event), and resell tickets if he wants to
sell them to other users (with the guarantee that he can't sell at a higher
price than the original). All of these use cases are important to give the user
the flexibility to manage his tickets and have the freedom to do what he wants
with them, within the system's rules, of course

\begin{figure}[H]
    \includegraphics[width=\textwidth/2]{User use cases.png}
    \centering
    \caption{User use cases}
    \label{fig:user_use_cases}
\end{figure}
