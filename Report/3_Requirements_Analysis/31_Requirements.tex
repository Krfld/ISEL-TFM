\section{Requirements}
\label{sec:requirements}

This section outlines the system requirements, categorized into functional and
non-functional requirements. The functional requirements, described in Section
\ref{subsec:functional_requirements}, define the specific features and
behaviors the system must support. In contrast, the non-functional
requirements, covered in Section \ref{subsec:non_functional_requirements},
address aspects like performance, security, and user experience.

\subsection{Functional Requirements}
\label{subsec:functional_requirements}

Table \ref{tab:functional_requirements} lists the functional requirements along
with their respective descriptions, outlining the core functionalities that the
system must implement.

\begin{table}[H]
    \centering
    \begin{tabularx}{\textwidth}{lX}
        \hline
        \textbf{Requirement}       & \textbf{Description}                                                                                                              \\
        \hline
        Wallet software            & The system must connect to a wallet software to interact with the blockchain, enabling the signing of transactions.               \\
        \hline
        Smart contract interaction & The system must interact with the smart contract to display event-related information and update the status of events or tickets. \\
        \hline
        File storage system        & The system must be able to upload the NFTs metadata to a file storage system, to store and retrieve them.                         \\
        \hline
    \end{tabularx}
    \caption{Functional Requirements}
    \label{tab:functional_requirements}
\end{table}

From this table, we can see that the system's core functionalities revolve around the interaction with the blockchain. The system must connect to a wallet software to sign transactions, interact with the smart contract to manage events and tickets, and store NFT metadata in a file storage system. These requirements are essential for the system to function as intended and provide a seamless experience for users.

\subsection{Non-Functional Requirements}
\label{subsec:non_functional_requirements}

Table \ref{tab:non_functional_requirements} presents the non-functional
requirements, defining the system's performance, security, scalability, and
overall quality to ensure an optimal user experience.

\begin{table}[H]
    \centering
    \begin{tabularx}{\textwidth}{lX}
        \hline
        \textbf{Requirement} & \textbf{Description}                                                                                     \\
        \hline
        Scalable             & The system must scale to accommodate a large number of users and events without performance degradation. \\
        \hline
        Low Fees             & The system must minimize operational fees to provide cost-efficient transactions.                        \\
        \hline
        Fast                 & The system must ensure fast processing times to offer a seamless experience for users during events.     \\
        \hline
        Secure               & The system must ensure security to prevent fraud or unauthorized access.                                 \\
        \hline
        User-Friendly        & The system must be intuitive and easy to use, facilitating the buying and selling of tickets.            \\
        \hline
    \end{tabularx}
    \caption{Non-Functional Requirements}
    \label{tab:non_functional_requirements}
\end{table}

The requirements for scalability, low fees, and speed are largely influenced by
the choice of the blockchain network. As discussed in Section
\ref{subsec:networks}, each blockchain network offers different
characteristics, and selecting the right one is crucial for meeting these
requirements. Meanwhile, security and user-friendliness are more dependent on
the system's design and implementation. The system must be engineered to
prevent fraud and ensure a smooth user experience.
