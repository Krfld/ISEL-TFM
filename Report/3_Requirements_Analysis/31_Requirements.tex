\section{Requirements}
\label{sec:requirements}

[brief introduction of the section]

This Section shows the requirements for the system, divided into functional and
non-functional requirements. The functional requirements, in the Section
\ref{subsec:functional_requirements}, are the features that the system has to
fulfill related to the functionality itself while the non-functional
requirements, in the Section \ref{subsec:non_functional_requirements}, are
related to other aspects like performance or security.

\subsection{Functional Requirements}
\label{subsec:functional_requirements}

In the Table \ref{tab:functional_requirements} there are the functional
requirements and we can see their respective description on the right side of
the table.

\begin{table}[H]
    \begin{tabularx}{\textwidth}{lX}
        \hline
        \textbf{Requirement}             & \textbf{Description}                                                                                                                                                                                                                                               \\
        \hline
        Connect to a wallet software     & The system must be able to connect                                                                                                      \newline to a wallet software to interact with the blockchain and to allow the validation of the ownership of the tickets. \\
        \hline
        Interact with the smart contract & The system has to interact with the smart contract to display the necessary information and to update the state of any event or ticket.                                                                                                                            \\
        \hline
        Add NFTs metadata to the IPFS    & The system must be able to add the NFTs metadata to the InterPlanetary File System (IPFS) to store that data offchain.                                                                                                                                             \\
        \hline
    \end{tabularx}
    \caption{Functional Requirements}
    \label{tab:functional_requirements}
\end{table}

\subsection{Non-Functional Requirements}
\label{subsec:non_functional_requirements}

For the non-functional requirements, the Table
\ref{tab:non_functional_requirements} shows the necessary criteria for the system to perform optimally and meet quality standards.

\begin{table}[H]
    \begin{tabularx}{\textwidth}{lX}
        \hline
        \textbf{Requirement} & \textbf{Description}                                                               \\
        \hline
        Secure               & The system must be secure to avoid any kind of fraud.                              \\
        \hline
        Low fees             & The system must have low fees for any kind of operation.                           \\
        \hline
        Scalable             & The system must be able to handle a large number of users and events.              \\
        \hline
        Fast                 & The system must be fast to allow events to have the smoothest experience possible. \\
        \hline
        User-friendly        & The system must be user-friendly to allow users to easily buy and sell tickets.    \\
        \hline
    \end{tabularx}
    \caption{Non-Functional Requirements}
    \label{tab:non_functional_requirements}
\end{table}

The scalable, low fees and fast requirements are essentially associated with
the blockchain network choice. As we saw in the Section \ref{subsec:networks},
different networks have different characteristics and we need to choose the one
that best fits our needs. The secure and user-friendly requirements are
associated with the systems design and implementation. We need to design the
system in a way that is secure, to avoid any kind of fraud, and have a smooth
experience for the users.
