\chapter{Gestão de Acrónimos}
\label{ch:ManAcronym}

O \LaTeX permite gerir acrónimos de forma automática. O primeio passo consiste em criar uma lista de acrónimos.
A lista de acrónimos é um ficheiro de texto com entradas do tipo:

\begin{verbatim}
	\newacronym{usb}{USB}{Universal Serial Bus}
\end{verbatim}


Para utilizar um acrónimo utiliza-se a expressão \Verb|\gls{nome}|. No primeira utilização o acrónimo é expandido 
e nas seguintes utilizações aparece apenas a sigla do acrónimo. Por exemplo, \gls{usb} e da segunda 
referência \gls{usb}.

No entanto, existem situações em pode ser necessário ter um controlo mais ``fino" para a expansão dos acrónimos. 
Para tal pode-se utilizar uma das seguintes expressões \LaTeX:

\begin{itemize}
	\item \Verb|\acrlong{nome}| -- para obter a definição do acrónimo
	\item \Verb|\acrshort{nome}| -- para obter a sigla do acrónimo
	\item \Verb|\acrfull{nome}| -- para obter a definição e sigla do acrónimo
\end{itemize}

No caso do acrónimo \textbf{usb} a utilização das expressões anteriores seria:

\begin{itemize}
	\item \acrlong{usb} -- para obter a definição do acrónimo
	\item \acrshort{usb} -- para obter a sigla do acrónimo
	\item \acrfull{usb} -- para obter a definição e sigla do acrónimo
\end{itemize}

Outros acrónimos que estão definidos são \gls{asv} e \gls{usv}. Que podem ser referidos várias vezes no texto: \gls{asv} e \gls{usv}.

