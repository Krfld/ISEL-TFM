\chapter{Linguagem Java}
\label{ch:LangJava}

A versão do exemplo \textbf{Hello World}, localizado num ficheiro e escrito na linguagem Java, é apresentada na Listagem \ref{lst:Java}.

\lstinputlisting[
	language=Java,
	caption=Hello World em \texttt{Java},
	label=lst:Java,
	%firstline=5
]
{Hello.java}

A Listagem \ref{lst:JavaVer2} apresenta um segundo exexmplo de uma listagem que apenas mostra parte de um ficheiro, nomeadamente a função \texttt{main(String[] args)}. É igualmente possível observar que quando uma linha de código sai das 
margens do texto é inserida de forma ``automática'' uma quebra de linha.

\lstinputlisting[
	language=Java,
	caption=Hello World em \texttt{Java} com recolha de nome,
	label=lst:JavaVer2,
	linerange={6-15}
]
{Hello-Ver2.java}

A listagem \ref{lst:JavaVer3} apresenta todo o programa mas exclui as linhas correspondentes aos \texttt{import}.

\lstinputlisting[
	language=Java,
	caption=Hello World em \texttt{Java} com recolha de nome - Programa completo,
	label=lst:JavaVer3,
	firstline=5
]
{Hello-Ver2.java}
